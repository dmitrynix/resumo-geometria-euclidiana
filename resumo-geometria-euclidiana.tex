\documentclass[11pt]{article}

\usepackage[brazil]{babel}
\usepackage[utf8]{inputenc}
\usepackage[T1]{fontenc}
\usepackage{gensymb}

\author{Dmitry Rocha}
\title{Resumo Geometria Euclidiana}

% From: http://www.wikihow.com/Write-a-Resume-in-LaTeX .
\topmargin=0.0in %length of margin at the top of the page (1 inch added by default)
\oddsidemargin=0.0in %length of margin on sides for odd pages
\evensidemargin=0in %length of margin on sides for even pages
\textwidth=6.5in %How wide you want your text to be
\marginparwidth=0.5in
\headheight=0pt %1in margins at top and bottom (1 inch is added to this value by default)
\headsep=0pt %Increase to increase white space in between headers and the top of the page
\textheight=10.0in %How tall the text body is allowed to be on each page

\begin{document}

\maketitle

\pagenumbering{arabic}

\section{Axioma de Incidência e de Ordem}

\noindent{\Large \bf I. Axioma de Incidência}

\begin{description}
  \item[Axioma I$_{1}$] Existem pontos que pertencem ou não a reta;
  \item[Axioma I$_{2}$] Dados dois pontos existe uma única reta que os contém;
  \item[Proposição 1.1] Duas retas distintas ou não se interceptam num único
    ponto.
\end{description}

\begin{quote}
  Intersecção de reta: ou é um único ponto ou é vazio
\end{quote}

\noindent{\Large \bf II. Axioma de Ordem}

\begin{description}
  \item[Axioma II$_{1}$] Dados 3 pontos distintos de uma reta um e apenas
    um deles localiza-se entre os outros dois.
  \item[Definição 1.2] O conjunto constituído por dois pontos $A$ e $B$ e por
    todos os pontos que se encontram entre $A$ e $B$ é chamado de
    \textbf{segmento de reta} $AB$. Os pontos $A$ e $B$ são denominados
    extremos do segmento.
  \item[Definição 1.3] Se $A$ e $B$ são pontos distintos, o conjunto
    constituído pelos pontos do segmento $AB$ e por todos os pontos $C$ tais
    que
    $B$ encontra-se entre $A$ e $C$, é chamado de \textbf{semi-reta} de
    origem $A$ contendo o ponto $B$, representado por $S_{AB}$. O ponto $A$ é
    denominado origem da semi-reta $S_{AB}$.
  \item[Proposição 1.4] Para as semi-retas determinada por dois pontos $A$ e
    $B$ temos:

  \begin{enumerate} \itemsep1pt \parskip0pt \parsep0pt
    \item $S_{AB} \cup S_{BA}$ é a reta determinada por $A$ e $B$;
    \item $S_{AB} \cap S_{AB} = AB$.
  \end{enumerate}

  \item[Axioma II$_{2}$] Dados dois pontos distintos $A$ e $B$ sempre existem
    um ponto $C$ entre $A$ e $B$ e um ponto $D$ tal que $B$ está entre $A$ e
    $D$.

  \begin{quote}
    Quaisquer dois pontos existem uma infinidade de pontos entre eles
  \end{quote}

  \item[Definição 1.5] Sejam $m$ uma reta e $A$ um ponto que não pertence a
    $m$. O conjunto constituído pelos pontos de $m$ e por todos os pontos $B$
    tais que $A$ e $B$ estão num mesmo lado da reta $m$ é chamado de
    \textbf{semiplano} determinado por $m$ contendo $A$, representado por
    $P_{mA}$.

  \item[Aximoma II$_{3}$] Uma reta $m$ determina exatamente dois semiplanos
    distintos cuja interseção é a reta $m$.
\end{description}

\line(1,0){470}

\begin{description}
  \item[Ponto] $A$, $B$, $C$, \ldots

  \item[Reta] $a$, $b$, $c$, \ldots

  \item[Segmento de reta]
  Dados dois pontos distintos $A$ e $B$ numa reta $r$, o conjunto de todos os
  pontos de $r$, entre $A$ e $B$ é dito segmento de reta. Exemplo: $AB$.

  \item[Semi-reta]
  Dados uma reta $r$ e um ponto $P$ sobre $r$, cada uma das partes de $r$,
  constituídas pelo ponto $P$ e todos os outros pontos de $r$ que estão de um
  mesmo lado do ponto $O$, é dita semireta. Exemplo: $S_{AB}$.

  \item[Semiplano]
  Uma reta $r$ num plano $\alpha$ , divide o plano em duas partes, cuja
  interseção é $r$. Cada uma dessas partes é dita semiplano.
\end{description}

\section{Axiomas Sobre Medição de Segmentos}

\begin{description}
  \item[Aximoma III$_{1}$] Todo par de pontos do plano corresponde um
    número\footnote{Coordenada} maior ou igual a zero. Zero se os pontos forem
    coincidentes.

  \item[Aximoma III$_{2}$] Os pontos de uma reta podem ser colocados em
    correspondência biunívoca com os números reais, de modo que a diferença
    entre estes números meça a distância entre os pontos correspondentes.
    \emph{Régua infinita}.

  \item[Aximoma III$_{3}$] Se o ponto $C$ encontra-se entre $A$ e $B$ então:
    $\overline{AC} + \overline{CB} = \overline{AB}$.

  \item[Proposição 2.1] Se, numa semi-reta $S_{AB}$, considerarmos um segmento
    $AC$ com $\overline{AC} < \overline{AB}$, então o ponto $C$ estará entre
    $A$ e $B$.

  \item[Definição 2.3] Ponto médio do segmento $AB$ a um ponto $C$ deste
    segmento tal que $\overline{AC} = \overline{CB}$.

  \item[Teorema 2.4] Um segmento tem exatamente um ponto médio.
\end{description}

\section{Axiomas sobre Medição de Ângulos}

\begin{description}
  \item[Definição 3.1] Ângulo: figura formada por duas semi-retas com mesma
    origem.

  \item[Axioma III$_{4}$] Todo ângulo tem uma medida maior ou igual a zero. A
    medida de um ângulo é zero se e somente se ele é constituiído por duas
    semi-retas coincidentes.

  \item[Definição 3.2] Diremos que uma semi-reta divide um semi-plano se ela
    estiver contida no semi-plano e sua origem for um ponto da reta que o
    determina.

  \item[Axioma III$_{5}$] É possível colocar (correspondência biunívoca)$^1$
    os números reais entre zero e 180 e as semi-retas da mesma origem que
    dividem um dado semi-plano, de modo que a diferença entre estes números
    seja a medida do ângulo formado pelas semi-retas correspondentes.

  \item[Axioma III$_6$] Se uma semi-reta $S_{OC}$ divide um ângulo $A\hat{O}B$,
    então $A\hat{O}B = A\hat{O}C + C\hat{O}B$
\end{description}

\line(1,0){470}

\begin{description}
  \item[Ângulo] Notação: $C\hat{A}B$, $\alpha$, $\hat{A}$.
  \item[Ângulo raso] igual a $180\degree$.
  \item[Ângulo reto] igual a $90\degree$.
  \item[Ângulos suplementares] adjacentes com soma $180\degree$.
  \item[Ângulos opostos pelo vértice (OPV)] são iguais.
\end{description}

\section{Congruência}

\begin{description}
  \item[Definição 4.1] Dois segmentos $AB$ e $CD$ são congruentes quando
    $\overline{AB} = \overline{CD}$; dois ângulos $\hat{A}$ e $\hat{B}$
    congruentes se têm a mesma medida.

  \item[Definição 4.2] Dois triângulos são congruentes se for possível
    estabelecer uma correspondência biunívoca entre seus vértices de modo que
    lados e ângulos correspondentes sejam congruentes.

  \item[Axioma IV] Dados dois triângulos $ABC$ e $EFG$, se $AB = EF$, $AC = EG$
    e $\hat{A} = \hat{E}$ então são congruentes.
    \textbf{1º caso de congruência de triângulos - LAL}.

  \item[Teorema 4.3]: Dados dois triângulos $ABC$ e $EFG$, se $AB = EF$,
    $\hat{A} = \hat{E}$ e $\hat{B} = \hat{F}$ então são congruentes.
    \textbf{2º caso de congruência de triângulos - ALA}.

  \item[Definição 4.4] Um triângulo é dito \textbf{isósceles} se tem dois lados
    congruentes. Estes lados chamados de laterais e o terceiro a base.

  \item[Proposição 4.5] Num triângulo isósceles os ângulos da base são
    congruentes.

  \item[Proposição 4.6] Se num triângulo tem-se dois ângulos congruentes,
    então o triângulo é isósceles.

  \item[Definição 4.7] Seja um triângulo $ABC$ um triângulo e seja $D$ um ponto
    da reta que contém $B$ e $C$. O Segmento $AD$ chama-se \textbf{mediada}
    do triângulo relativo ao lado $BC$, se $D$ for o ponto médio de $BC$.

    O segmento $AD$ chama-se \textbf{bissetriz} do ângulo $\hat{A}$ se a
    semi-reta $S_{AD}$ divide o ângulo $C\hat{A}B$ em dois ângulos
    congruentes, isto é, se $C\hat{A}D = D\hat{A}B$.

    O segmento $AD$ chama-se altura do triângulo relativamente ao lado $BC$, se
    $AD$ for perpendicular a reta que contém $B$ e $C$.

  \item[Proposição 4.8] Num triângulo isósceles a mediana relativamente a base
    é também bissetriz e altura.

  \item[Teorema 4.9] Se dois triângulos tem três lados correspondentes
    congruentes então os triângulos são congruentes.
    \textbf{3º caso de congruência de triângulos - LLL}.
\end{description}

\line(1,0){470}

\begin{description}
  \item[$AB = CD$] $AB$ é congruente a $CD$
  \item[$\overline{A} = \overline{B}$] $\overline{A}$ é congruente a
    $\overline{B}$
\end{description}

\section{O Teorema do Ângulo Externo e Suas Consequências}

\begin{description}
  \item[Teorema 5.2 (Ângulo Externo)] Todo ângulo externo de um triângulo
    mede mais do que qualquer dos ângulos internos a ele não adjacentes.

  \item[Proposição 5.3] A soma das medidas de quaisquer dois ângulos
    internos de um triângulo é menor do que $180\degree$.

  \item[Corolário 5.4] Todo ângulo possui pelo menos dois ângulos internos
    agudos.

  \item[Corolário 5.5] Se duas retas distintas $m$ e $n$ são perpendiculares
    a uma terceira, então $m$ e $n$ não se interceptam.

  \item[Definição 5.6] Duas retas que não se interceptam são ditas paralelas.

  \item[Proposição 5.7] Por um ponto fora de uma reta passa uma e somente uma
    reta dada.

  \item[Proposição 5.8] Se dois lados de um triângulo não são congruentes
    então seus ângulos opostos não são congruentes e o maior ângulo é oposto
    ao maior lado.

  \item[Teorema 5.10] Em todo triânglo, a soma dos comprimentos de dois lados
    é maior do que o comprimento do terceiro lado.

\end{description}

\end{document}
