\documentclass[11pt]{article}

\usepackage[brazil]{babel}
\usepackage[utf8]{inputenc}
\usepackage[T1]{fontenc}

\author{Dmitry Rocha}
\title{Resumo Geometria Euclidiana}

% From: http://www.wikihow.com/Write-a-Resume-in-LaTeX .
\topmargin=0.0in %length of margin at the top of the page (1 inch added by default)
\oddsidemargin=0.0in %length of margin on sides for odd pages
\evensidemargin=0in %length of margin on sides for even pages
\textwidth=6.5in %How wide you want your text to be
\marginparwidth=0.5in
\headheight=0pt %1in margins at top and bottom (1 inch is added to this value by default)
\headsep=0pt %Increase to increase white space in between headers and the top of the page
\textheight=9.0in %How tall the text body is allowed to be on each page

\begin{document}

\maketitle

\pagenumbering{arabic}

\noindent{\Large \bf I. Axioma de Incidência}

\emph{I$_{1}$}: Existem pontos que pertencem ou não a reta;

\emph{I$_{2}$}: Dados dois pontos existe uma única reta que os contém;

\emph{Proposição 1.1}: Duas retas distintas ou não se interceptam num único
ponto.

\begin{quote}
  Intersecção de reta: ou é um único ponto ou é vazio
\end{quote}

Pontos: $A$, $B$, $C$, \ldots

Retas: $a$, $b$, $c$, \ldots

\noindent{\Large \bf II. Axioma de Ordem}

\emph{II$_{1}$}. Dados 3 pontos distintos de uma reta um e apenas um deles
localiza-se entre os outros dois.

\emph{Definição 1.2} O conjunto constituído por dois pontos $A$ e $B$ e por
todos os pontos que se encontram entre $A$ e $B$ é chamado de \textbf{segmento de
reta} $AB$. Os pontos $A$ e $B$ são denominados extremos do segmento.

\emph{Definição 1.3} Se $A$ e $B$ são pontos distintos, o conjunto
constituído pelos pontos do segmento $AB$ e por todos os pontos $C$ tais que
$B$ encontra-se entre $A$ e $C$, é chamado de \textbf{semi-reta} de origem $A$
contendo o ponto $B$, representado por $S_{AB}$. O ponto $A$ é denominado
origem da semi-reta $S_{AB}$.

\emph{Proposição 1.4} Para as semi-retas determinada por dois pontos $A$ e
$B$ temos:

\begin{enumerate} \itemsep1pt \parskip0pt \parsep0pt
\item $S_{AB} \cup S_{BA}$ é a reta determinada por $A$ e $B$;
\item $S_{AB} \cap S_{AB} = AB$.
\end{enumerate}

\emph{II$_{2}$} Dados dois pontos distintos $A$ e $B$ sempre existem um ponto $C$
entre $A$ e $B$ e um ponto $D$ tal que $B$ está entre $A$ e $D$.

\begin{quote}
  Quaisquer dois pontos existem uma infinidade de pontos
\end{quote}

\emph{Definição 1.5} Sejam $m$ uma reta e $A$ um ponto que não pertence a
$m$. O conjunto constituído pelos pontos de $m$ e por todos os pontos $B$ tais
que $A$ e $B$ estão num mesmo lado da reta $m$ é chamado de semiplano
determinado por $m$ contendo $A$, representado por $P_{mA}$.

\emph{II$_{3}$} Uma reta $m$ determina exatamente dois semiplanos distintos
cuja interseção é a reta $m$.

\line(1,0){470}

\begin{description}
\item[Segmento de reta]
Dados dois pontos distintos $A$ e $B$ numa reta $r$, o conjunto de todos os
pontos de $r$, entre $A$ e $B$ é dito segmento de reta. Exemplo: $AB$.

\item[Semi-reta]
Dados uma reta $r$ e um ponto $P$ sobre $r$, cada uma das partes de $r$,
constituídas pelo ponto $P$ e todos os outros pontos de $r$ que estão de um
mesmo lado do ponto $O$, é dita semireta. Exemplo: $S_{AB}$.

\item[Semiplano]
Uma reta $r$ num plano $\alpha$ , divide o plano em duas partes, cuja
interseção é $r$. Cada uma dessas partes é dita semiplano.
\end{description}

\end{document}
